\section{Development Environment (\textit{Yiwei})}

\subsection{Platforms}

The team used various development environments to code different sections
of the compiler. Members used whichever platform they were most comfortable
and familiar with. Our code was run and tested on Windows, Mac, Linux, and
Unix. The full environment consists of Tiny OS \cite{dunkels:2004a} and
Fennec Fox installations, among other things as software, and the sensor
motes as the most important hardware. However, for an experimenter to try
out our compiler the aforementioned software and hardware are not needed;
the compiler will produce the output nesC code independently. It just that
the experimenter will not be able to see the generated code running. That
was also the case for our team since we do not have the capability to
equip a full system for each team member. Still we are able to develop the code in individual machines throughout the project.

\subsection{Programming Tools}

In order to develop the Swift Fox language, Flex and Bison were used to
create the lexer and parser. Any other code in the lexer and parser were
written in C. Due to the original Fennec Fox that Swift Fox is intended to 
be built on top of, our target language is nesC \cite{gay:2003}. Since nesC
is a close variant of C, we chose to use C code to do code generation into 
nesC, as well as error checking. Code for defining the tree nodes,
traversal, code generation and error generation have their individual
\texttt{.h} and \texttt{.c} files and are linked together using the
corresponding makefiles. The Makefile compiles all necessary \texttt{.c}
files and creates a Swift Fox program. In addition, simple shell scripts
were created to compile, run and clean the \texttt{.o} files at the end of 
tests. The majority of the functions required the standard C library
as well as the standard I/O library. For code editing, individual members
used Eclipse, Notepad++, or Vi.

\subsection{Management Tools}

To synchronize code in a distributed developing environment, the team used 
Subversion (SVN) as a repository for code (and also documentation). In
combination with SVN, Trac was used to keep track of deadlines,
documentation, and tickets. Other alternatives such as DropBox, Google Docs
were considered but since there is a strong likelihood of future code
development of Swift Fox, code needed to be public and a website needed to 
be created for both the posting of code and extensive documentation for
future developers or users. In the end, a web server running Trac was set
up 

Trac is a free open source web-development tool to assist with project
management. It allows maintaining all aspects of the project in one place, 
like:
\begin{itemize}
	\item wiki. Common definition, role assignment, meeting minutes,
	tools used, and so forth
	\item timelines. Keeping track of who did what and when, ticketing
	assignment and progress watching of bite-sized tasks
	\item code plus document repository. Using track we are able to
	look at the code resided in SVN repository right from browser
	window, which is very convenient. 
\end{itemize}

Nevertheless, there were some synchronization problems with SVN. While Trac
reduced some synchronization issues, email alerts as to which member was
currently editing which part of code were still frequent. Also, certain
files and folder needed to be copied and edited locally in order to prevent
corruption of the repository. By the end of the project, most of these
issues were eliminated.
