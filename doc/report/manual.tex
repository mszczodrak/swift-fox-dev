\section{Introduction}
\label{sec:manual_introduction}

This manual describes the first draft of Swift Fox language as specified by
the authors on \today. In the following sections we define in detail the
basic constructs used in Swift Fox. Special consideration have been given in
making this manual reader-friendly and comprehensible, but at the same time
formal enough so as to capture the the basic concepts that Swift Fox 
employs.

\

\hangindent=4cm
\hangafter=0
\small
\noindent
Commentary material is indented and written in smaller type, similar to
this excerpt.
\normalsize

\section{Lexical Conventions}
\label{sec:lexical_conventions}

In the Swift Fox language, a program consists of one or more
\textit{translation units} stored into different files. The translation
process consists of multiple phases, which are described in great detail in 
\cite{marcin:tutorial}. In the following section, we present the primitive 
constructs that are used throughout the first phase of the translation,
namely the \textit{lexical analysis}. During that phase, the source code of
a Swift Fox program is considered to be a stream of characters that is fed
into the lexical analyzer (lexer). Subsequently, the lexer\footnote{Also
known as \textit{scanner}.} groups sequences of characters together and
identifies \textit{tokens}. Each token is a pair of a \texttt{name} and a
\texttt{value}; the value corresponds to a particular \textit{lexeme} that 
is identified by the regular expression for the corresponding token,
whereas the name is merely an identifier for abstracting the set of lexemes
(\textit{i.e.,} referring to the language of lexemes that are described by 
the same regular expression) \cite{aho:2007}.

\

$\textit{token:} <\texttt{token\_name}, \texttt{token\_value}>$
(\textit{e.g.,} $<\texttt{identifier}, \texttt{blink}>$)

\

\hangindent=4cm
\hangafter=0
\small
\noindent
Appendix \ref{sec:lex} includes the definition of a Swift Fox lexical
analyzer for the \texttt{flex} lexer-generator \cite{lesk:1990,paxson:2010}.
The reference implementation of the Swift Fox compiler uses the
\texttt{flex} lexer-generator and the \texttt{YACC} parser-generator,
\cite{johnson:1975,bison:2010}, however this manuscript provides
(hopefully) all possible details for implementing a Swift Fox compiler in
every available compiler-compiler tool.
\normalsize

\subsection{Tokens}

\textit{Tokens} in Swift Fox, similarly to every other language, are
composed by sequences of characters. The \textit{lexer} (\textit{i.e.,}
lexical analyzer) produces four classes of tokens:
\begin{itemize}
	\item \texttt{identifiers}
	\item \texttt{keywords}
	\item \texttt{constants}
	\item \texttt{operators}
\end{itemize}

Additionally, the \textit{whitespace} character class is used in order to
separate the various tokens. The following characters are considered
whitespace and along with comments are ignored throughout the rest phases
of the translation.

\begin{itemize}
	\item \texttt{space}
	\item \texttt{horizontal tab}
	\item \texttt{vertical tab}
	\item \texttt{form feed}
	\item \texttt{carriage return}
\end{itemize}

Notice, however, that \texttt{newlines} are not considered whitespace and
they have a special meaning in Swift Fox (see Section
\ref{sec:separators}).

\

\hangindent=4cm
\hangafter=0
\small
\noindent
Swift Fox is currently under development. Therefore, there might be some
inconsistency between the formal description of the tokens and the
corresponding \texttt{Lex} definition in Appendix \ref{sec:lex}.
\normalsize

\subsection{Comments}

\textit{Comments} are introduced by character \texttt{\#} and terminate
with a newline. Everything between the comment character and the newline
character is considered to be a comment, and therefore it is ignored.
Comments must start on a new line, and can only be preceded by a delimiter,
but not a line of a Swift Fox code.

\subsection{Identifiers}

\textit{Identifiers} in Swift Fox are sequences of letters, digits, and
character ``\texttt{\_}''. They are used in order to \textit{name} specific
instances of the primitive types that the language provides (see Section
\ref{sec:types}) and, hence, they are considered to be variables.
Identifiers can be of arbitrary length but not zero (\textit{i.e.,} there
is no such thing as the ``empty'' identifier). Moreover, the first
character of an identifier cannot be a number; it must be either a
letter, or character ``\texttt{\_}''. Finally, identifiers are case
sensitive and have a number of characters that are considered to be
significant.

\

\hangindent=4cm
\hangafter=0
\small
\noindent
Notice that the number of significant characters in identifiers depends on 
the compiler implementation. The reference implementation provided by the
authors supports up to 127 significant characters.
\normalsize

\

Identifiers in Swift Fox are only used for naming objects (\textit{e.g.,}
applications, networks, sources, configurations, event-conditions) and
referring to them. The binding between a name and the corresponding object
is performed during the declaration of the identifier and remains the same 
for the rest of the program. For example, Swift Fox does not allow a user 
to change a configuration variable and assign a different configuration 
to it. The \textit{scope} of all identifiers is, therefore, the same and
it global (for a particular program). Similarly, the \textit{lifetime} of 
all identifiers is the same and it is equal to the lifetime of the Swift 
Fox program. 

\

\hangindent=4cm
\hangafter=0
\small
\noindent
Objects in Swift Fox are not created and disposed \textit{dynamically} (or
\textit{automatically}) at runtime.
\normalsize

\subsection{Keywords}

Swift Fox uses a set of special-meaning identifiers, namely
\textit{keywords}, which cannot be used otherwise. Table
\ref{tbl:keywords}, below, illustrates all the ``reserved'' words that are 
used in Swift Fox.

\begin{table}[!h]
	\begin{center}
	\begin{tabular}{| c | c | c |}
	\hline
	\texttt{use} & \texttt{application} & \texttt{network}	\\
	\hline
	\texttt{source} & \texttt{configuration} & 
	\texttt{event-condition}					\\
	\hline
	\texttt{from} & \texttt{goto} & \texttt{when}			\\
	\hline
	\texttt{nothing} & \texttt{any} & \texttt{once}			\\
	\hline
	\texttt{and} & \texttt{or} & \texttt{not} 			\\
	\hline
	& \texttt{start} &	\\
	\hline
	\end{tabular}
	\end{center}
	\caption{Swift Fox keywords.}
	\label{tbl:keywords}
\end{table}

\

\hangindent=4cm
\hangafter=0
\small
\noindent
All keywords have special meaning for the Swift Fox translation procedure
and therefore cannot be used as identifiers; one cannot name a
configuration or an event-condition as ``network'', or ``goto''.
\normalsize

\subsection{Constants}

Currently only four different kinds of \textit{constants} are supported.
Temperature, time, integer constants, and string literals. Temperature and
time constants are comprised of digit sequences that resemble a decimal
value, followed by the corresponding suffix that denotes the scale. The
allowed suffixes for temperature are ``\texttt{C}'' and ``\texttt{F}'' and
they denote the Celsius and Fahrenheit scale respectively. Similarly, the
time-allowed suffixes are ``\texttt{msec}'', ``\texttt{sec}'',
``\texttt{min}'', ``\texttt{hr}''. On the other hand, integer constants are
not allowed to have a suffix; they are merely comprised of digit sequences 
that resemble a decimal value. Finally, Swift Fox also supports string
literals, which are essentially sequences of characters.

\

\hangindent=4cm
\hangafter=0
\small
\noindent
Typically, string constants are used in order to denote path elements on
the definition of applications, network protocols, and sources. They are
semantically compatible with the definition of C strings.
\normalsize


\section{Syntax Notation}
\label{sec:syntax_notation}

In this section we define the syntactic constructs that are used in Swift
Fox language. The \textit{syntactic analysis} is the bulk part of the
second phase of the translation procedure \cite{aho:2007}. During that
phase, a stream of tokens, which is provided by the lexer, is checked for
adherence to Swift Fox rules. In other words, the source code is tested for
conformity to the syntactic rules that are defined by the formal grammar of
the language. The result of this procedure is an \textit{annotated parse 
tree} of the program along with a special data structure that keeps
information about each identifier (\textit{i.e.,} the \textit{symbol
table}).

\

\hangindent=4cm
\hangafter=0
\small
\noindent
Appendix \ref{sec:yacc} includes the definition of a Swift Fox parser. The
reference implementation of the Swift Fox compiler uses the YACC (Bison)
parser-generator \cite{johnson:1975,bison:2010}. However, this manuscript
provides all possible details for implementing a Swift Fox compiler with
every available compiler-compiler.
\normalsize

\subsection{Types}
\label{sec:types}

Swift Fox is a \textit{strongly typed} language \cite{cardelli:1991} where
type checking occurs only at compile time. The simplistic nature of Swift
Fox syntax prevents type intermixing without loss of expressibility. The
benefits of this approach are manifold. First, the language does not need
to involve additional complexity with runtime checks. Notice that the
execution environment of Swift Fox is severely constrained and any runtime 
type check will not only impact the performance of the system, but also
drain the available power supply \cite{marcin:whitepaper}. Second, it
allows strong guarantees about the runtime behavior of the program since
there are no explicit or implicit type conversions. Third, it guards
against evasions of the type system that can lead to unpredictable
behavior.

Swift Fox is also a \textit{static typed} language. This allows the optimal
selection of the storage needed for the various Swift Fox objects. This is,
again, of paramount importance for the over constrained application domain
of Swift Fox \cite{marcin:whitepaper}. The \textit{primitive} types that
define the building blocks of the language are the following:

\begin{itemize}
	\item \texttt{application}. This primitive type is used in order to
	define application components (\textit{e.g.,} Blink, Collector, or
	Picture) as introduced in \cite{marcin:tutorial}
	\item \texttt{network}. Similarly to the previous type, network
	is used in order to define network components (\textit{e.g.,}
	CTP or P2P-MultiHop)
	\item \texttt{source}. Defines event sources (\textit{e.g.,}
	temperature readings, timeouts)
	\item \texttt{configuration}. Configuration defines a specific
	binding among instances of the different classes of components
	that are provided by Fennec Fox \cite{marcin:tutorial})
	\item \texttt{event-condition}. Event-condition associates events
	with specific conditions (\textit{e.g.,} temperature $>$ 90 F)
\end{itemize}

\

\hangindent=4cm
\hangafter=0
\small
\noindent
Swift Fox does not have composite types nor does it allow the programmer to
define and use its own types. Moreover, it provides a set of predefined
values for application, network, and source that correspond to the basic
applications, network protocols, and sensors that are always available to a
system. This feature is essentially similar to the notion of C libraries
that provide additional functionality in user programs when needed.
\normalsize

\subsection{Operators}

The operators used in Swift Fox can be classified as \textit{relational},
\textit{logical}, or \textit{enumerative}. Relational operators are used
in order to define an event-condition and they evaluate to \texttt{true}
or \texttt{false}. Infix operators \texttt{<} (less), \texttt{>} (greater),
\texttt{<=} (less or equal), \texttt{>=} (greater or equal), and
\texttt{=} (equal) yield 0, if the relation that corresponds to a specific
condition is false, and 1 if it is true. The infix logical operators
\texttt{and} and \texttt{or} are used in order to combine two or more
event-conditions conjunctively or disjunctively, whereas the infix
\textit{comma} (\texttt{,}) operator is used among two or more
configurations in order to create a set from them. The \texttt{not} 
operator works as negation operator and takes
one argument of type \texttt{event}. The return value of the \texttt{not}
operator is an event which becomes true when the argument event is false.
The \texttt{and} operator takes two arguments of type \texttt{event} and
returns an event which is true when both argument events are true. The
\texttt{or} operator takes two arguments of type \texttt{event} and
returns two events that are true if either one of the argument events is
true. To improve readability, in every statement we allow to use maximum
one \texttt{or} operator. The number of \texttt{not} and \texttt{and}
operators is unlimited. Finally, the \texttt{not} operator has higher 
precedence than the operator \texttt{and}, and operator \texttt{or} has 
the lowest precedence. 

\subsection{Separators}
\label{sec:separators}

\textit{Separators} are special characters that are used in Swift Fox in
order to segregate various statements. Currently, Swift Fox uses only one
separator: the \texttt{newline} (\texttt{LF}). Hence, library declaration
statements, configuration and event-condition declaration statements, and
policy and initial-configuration statements (Section \ref{sec:statements}
and Appendix \ref{sec:yacc}) are separated from each other using newline
characters.

\subsection{Statements}
\label{sec:statements}

Swift Fox programs consist of sequences of \textit{statements}. There are
three different types of \textit{statements}. The first type includes
\textit{declaration statements} for applications, network protocols, and
event sources, as well as configurations and event-conditions (see Section 
\ref{sec:types}). The lines 154 -- 231 and 394 -- 453 in Appendix
\ref{sec:yacc} illustrate the syntax for that particular type. The second
type of statements is about \textit{policy definitions} (see lines 252 --
380 in Appendix \ref{sec:yacc}). The bulk part of a Swift Fox program is
typically made of such statements, since they capture the reconfiguration
strategy of the system. Finally, the last type of statement (actually it
is only one statement) is about declaring the initial configuration of the
system; similar to the main entry point of a C program, there is an initial
configuration for a Swift Fox policy.

\

\hangindent=4cm
\hangafter=0
\small
\noindent
Appendix \ref{sec:yacc} illustrates the syntax of the Swift Fox language
through a YACC grammar-definition. Without loss of generality, this should
give an indication of the syntactic rules of the language so as to
implement those in other compiler-compilers.
\normalsize

\

Notice that the order of the previous statements is important and fixed. A 
valid Swift Fox program cannot intermix different types of statements and
the first statements of a valid program should be application, network, or 
source declarations followed by configuration declarations. The
application, network, and source declarations are optional (\textit{i.e.,} 
they can be omitted) and are typically provided in the form of
a ``library''. Subsequently, there might be some event-condition
declaration statements followed by policy statements. Again, the
declaration of event-conditions and the definition of policies is not
mandatory. The final statement is always the initial configuration
statement and it is necessary in every valid Swift Fox program.

